\documentclass[a4paper]{scrartcl}
\usepackage[utf8]{inputenc}
\usepackage[english]{babel}
\usepackage{graphicx}
\usepackage{lastpage}
\usepackage{pgf}
\usepackage{wrapfig}
\usepackage{fancyvrb}
\usepackage{fancyhdr}

\usepackage{pdflscape}

\usepackage[font=footnotesize,labelfont=bf,skip=2pt]{caption}
\usepackage{hyperref}

% Code highligting
% \usepackage{minted}
\usepackage[outputdir=output/tex]{minted} % iom min makefile

\newenvironment{longlisting}{\captionsetup{type=listing}}{}
% \renewcommand\listoflistingscaption{Källkod....}
\renewcommand\listoflistingscaption{List of source codes}
\setmintedinline[sql]{breaklines=true,breakanywhere=true} % necessary for breakanywhere to work later on.

\pagestyle{fancy}

% Create header and footer
\headheight 27pt
\pagestyle{fancyplain}
\lhead{\footnotesize{Data Storage Paradigms, IV1351}}
\chead{\footnotesize{Project Report, Task 3}}
\rhead{}
\lfoot{}
\cfoot{\thepage\ (\pageref{LastPage})}
\rfoot{}


\title{Project Report, Task 3}
\subtitle{Data Storage Paradigms, IV1351}
\author{Vincent Ferrigan ferrigan@kth.se}
\date{\today}

\begin{document}
\maketitle
    
% \section*{Tips for Report Writing}
% \textbf{REMOVE THIS SECTION BEFORE SUBMITTING THE REPORT.}\\

% \noindent \textit{The target audience has exactly the same skills as the author,
% except they do not know anything at all about the specific application described
% in the report.} \\

% Consider the following:

% \begin{itemize}
%   \item \textbf{The report must be \textit{centered around the requirements}.
%   Which are they (Introduction), how did you work to meet them (Method), what is
%   the solution that meets them (Result), and how can you be sure they are met
%   (Discussion). This is the IMRaD method.} The requirements on the Introduction,
%   Method, Result and Discussion chapters are described below under each chapter.

%   \item Is spelling and grammar correct? Is spoken language avoided?

%   \item Does the report have a good structure with sections, subsections and
%   paragraphs?

%   \item Is the text clarified with images and/or other figures, and with links
%   to the code in your Git repository? Remember that all figures (images, tables,
%   graphs, code listings, etc) shall be numbered and have a short explaining
%   text.
% \end{itemize}

\section{Introduction}
The assignment involved creating 
\emph{Online Analytical Processing} (OLAP), \emph{Queries} and \emph{Views} 
on the Soundgood music school database. 
The purpose of these queries were to analyze the business and create reports.
The efficacy of one of the queries was analyzed using EXPLAIN ANALYZE and later
optimized by using join-operations instead of subqueries. 

The author of this report collaborated with
\emph{Elin Blomquist}.
% \textbf{This chapter tells \textit{what} are you going to do.} 

% Explain the task and the requirements on the solution. It's important to clearly
% state the requirements. \textit{Also specify which other student you worked with
% when solving the tasks, or if you worked alone.} 
\subsubsection*{GitHub-Repo}
\url{https://github.com/VincentFerrigan/kth-iv1351-data-storage-paradigms}

\section{Literature Study}
% This chapter must prove that you collected sufficient knowledge before starting
% development, instead of just hacking away without knowing how to complete a
% task. State what you have read and briefly summarize what you have learned.
The pre-recorded lecture on SQL, given by 
Paris Carbone, 
was reviewed, 
and chapters related to temporal database concepts and other
strategies for query processing in the main textbook
(Fundamentals of Database Systems) was studied. 
Aspects related to denormalization was briefly studied in the alternative
textbook (Database System Concepts).
were examined. 

Additionally, the document 
\href{https://canvas.kth.se/courses/43013/files/7104569?wrap=1}{tips-and-tricks-task3.pdf}
was studied.
Proper syntax related to procedures, functions, and similar topics was acquired
from both the PostgreSQL official documentation and the PostgreSQL Tutorial
website.

The project group also relied on the textbook Databasteknik, written by
Thomas Pardon-McCartny and Tore Risch to solve the task at hand.

\clearpage
\section{Method}
\label{sec:method}
% \textbf{This chapter tells \textit{how} you solved the task.}
% Explain how you worked when solving the tasks and how you evaluated that your
% solution met the requirements. \textit{Do not explain your solution and do not
% refer to code}, that belongs to the \textit{Result} chapter. More specific
% instructions for the content can be found under each task on the Project page in
% Canvas.

\subsection*{Tools}
The project group chose to use PostgreSQL as Database Management System (DMBS).
This report was also written in plain text mode -- \LaTeX.

All code was written in \emph{Visual Studio Code}.
Quick-fixes and editing was, however, done in \emph{Vim}. 
\emph{GIT} and \emph{GitHub} was used for version control.

\subsection*{The overall Work-flow}
The task at hand required, apart from the tools mentioned above,
pen and paper.
All expected output from the queries had to
initially be determined through manual calculations using the provided
\href{https://github.com/VincentFerrigan/kth-iv1351-data-storage-paradigms/blob/main/src/data/populate_tables.sql}{dummy data}.
This preliminary step involved manual computations on paper.

The development of the queries was an iterative process,
characterized by successive refinements based on continuous evaluations and adjustments
In other words, ''trail \& error'', constantly revisiting the
PostgreSQL official documentation (including their Tutorial website).

\pagebreak
\section{Result}
\label{sec:result}
The scripts and models for the entire project can be found on GitHub:

\subsubsection*{GitHub-Repo}
\url{https://github.com/VincentFerrigan/kth-iv1351-data-storage-paradigms}

\subsection*{Structured layout}
\label{subsec:struturedlayout}

\begin{itemize}
    \item \textbf{src/} (Source Scripts)
    \begin{itemize}
        \item \textbf{tables/}: Script for table creation.
        \href{https://github.com/VincentFerrigan/kth-iv1351-data-storage-paradigms/blob/main/src/tables/create_tables_soundgood_db.sql}{create\_tables\_soundgood\_db.sql}
        \item \textbf{functions/}: Script defining functions.
        \href{https://github.com/VincentFerrigan/kth-iv1351-data-storage-paradigms/blob/main/src/functions/functions.sql}{functions.sql}
        \item \textbf{procedures/}: Script for stored procedures.
        \href{https://github.com/VincentFerrigan/kth-iv1351-data-storage-paradigms/blob/main/src/procedures/procedures.sql}{procedures.sql}
        \item \textbf{data/}: Scripts for populating tables with data.
        \href{https://github.com/VincentFerrigan/kth-iv1351-data-storage-paradigms/blob/main/src/data/populate_tables.sql}{populate\_tables.sql}
        \item \textbf{analytics/:} Scrips for queries and analytics (task 3).
        \href{https://github.com/VincentFerrigan/kth-iv1351-data-storage-paradigms/blob/main/src/analytics/analytics.sql}{analytics.sql}
    \end{itemize}

    \item \textbf{diagrams/}
    \begin{itemize}
        \item Database schema diagrams and other architectural visuals.
    \end{itemize}

    \item \textbf{tex/} (Reports)
    \begin{itemize}
        \item \LaTeX files the seminar reports.
    \end{itemize}

    \item \textbf{outputs/} (e.g. EXPLAIN ANALYZE)
    \begin{itemize}
        \item Miscellaneous dB ouputs for analysis and manual tests.
    \end{itemize}

\end{itemize}

%\textbf{Root Directory:}
%\begin{itemize}
%  \item Include a \texttt{README.md} for an overview of the project.
%  \item A \texttt{.gitignore} file for Git version control.
%\end{itemize}

\subsection*{First Query}
    Views cannot have parameters, which is a problem when data
    should be given for a specific year.
    One approach is to manually filter the data using the
    WHERE clause in the SELECT queries.
    Another approach is to use a function.
    When called with a specific year, the function shown in listing ~\ref{listing:q1_func}, 
    will return a table
    summarizing the total number of given lessons, as well as counts for individual,
    group, and ensemble lessons for each month of that year -- using the data
    aggregated in the \mintinline{sql}{view_course_type_summary} found in
    listing ~\ref{listing:q1_view}. 
    The function essentially
    provides a yearly breakdown of session types per month.

    The output of this query is to be found in table ~\ref{table:q1}.

\begin{longlisting}
  \inputminted[
      label=Q1-Function,
      linenos=true,
      bgcolor=lightgray,
      firstline=15,
      lastline=26,
%        frame=single,
      fontsize=\footnotesize,
  ]{sql}{../../src/db/analytics/analytics.sql}
  \caption{
    Function for filtering and ordering output-table from view on given year.
    }
  \label{listing:q1_func}
\end{longlisting}

\begin{longlisting}
  \inputminted[
      label=Q1-VIEW,
      linenos=true,
      bgcolor=lightgray,
      firstline=15,
      lastline=26,
%        frame=single,
      fontsize=\footnotesize,
  ]{sql}{../../src/db/analytics/analytics.sql}
  \caption{Creating the database view for Query 1}
  \label{listing:q1_view}
\end{longlisting}

\begin{table}[h!]
\centering
\begin{tabular}{|l|l|l|l|l|}
\hline
Month & Total & Individual & Group & Ensemble \\ \hline
Jan   & 3     & 1          & 2     & 0        \\
Feb   & 5     & 2          & 3     & 0        \\
Mar   & 3     & 2          & 1     & 0        \\
Apr   & 7     & 5          & 0     & 2        \\
May   & 2     & 1          & 0     & 1        \\
Sep   & 2     & 1          & 0     & 1        \\
Oct   & 2     & 2          & 0     & 0        \\
Nov   & 2     & 2          & 0     & 0        \\
Dec   & 4     & 1          & 0     & 3        \\ \hline
\end{tabular}
\caption{Number of lessons given per moth during 2023. 
Output generated from the first query}
\label{table:q1}
\end{table}

\subsection*{Second Query}
When it is necessary to retrieve data from multiple tables within a database,
there exist two primary alternatives: \emph{subqueries} and \emph{joins}.
Subqueries involve nesting a query within another, whereas join operations
combine rows from various tables, ''joining'' them through a shared column or a
specific condition.

The project group chose to use both to solve the second query assignment
(see figure \ref{listing:q2_view_join} and \ref{listing:q2_view_sub}).
Their performance were compared
with \mintinline{sql}{EXPLAIN ANALYSE} (see subsection below). 

\begin{longlisting}
  \inputminted[
      label=Q2-VIEW with Join-operations,
      linenos=true,
      bgcolor=lightgray,
      firstline=96,
      lastline=111,
%        frame=single,
      fontsize=\footnotesize,
  ]{sql}{../../src/db/analytics/analytics.sql}
  \caption{Creating the database view for Query 2 with Join-operations}
  \label{listing:q2_view_join}
\end{longlisting}

\begin{longlisting}
  \inputminted[
      label=Q2-VIEW with Subqueries,
      linenos=true,
      bgcolor=lightgray,
      firstline=15,
      lastline=26,
%        frame=single,
      fontsize=\footnotesize,
  ]{sql}{../../src/db/analytics/analytics.sql}
  \caption{Creating the database view for Query 2 with subqueries}
  \label{listing:q2_view_sub}
\end{longlisting}

\begin{table}[h]
\centering
\begin{tabular}{|l|l|}
\hline
No. of Siblings & No. of Students \\ \hline
0               & 4               \\
1               & 14              \\
2               & 3               \\
5               & 6               \\ \hline
\end{tabular}
\caption{Number of students without siblings, one sibling, two etc.
Output generated from the first query}
\label{table:q2}
\end{table}

% Switch to landscape
\begin{landscape}
\subsubsection*{EXPLAIN ANALYZE}
The query plans in listing ~\ref{listing:q2_explain} 
show how PostgreSQL
processes the queries, including
operations like sorting, aggregation, joining, and scanning.
The metrics provide
insights into the efficiency of the queries, 
in terms of both planning and
execution time. 

\begin{longlisting}
  \inputminted[
      label=EXPLAIN ANALYZE,
      linenos=true,
      bgcolor=lightgray,
      firstline=1,
      lastline=78,
%        frame=single,
      fontsize=\footnotesize,
  ]{sql}{../../outputs/q2_explain_analyze.output}
  \caption{Comparing query plans and their performance for query 2}
  \label{listing:q2_explain}
\end{longlisting}

\end{landscape}
% Back to portrait

\clearpage
\subsection*{Third Query}
The function
\mintinline{sql}{fn_get_instructor_lesson_counts_above_threshold}
in listing \ref{listing:q3}
is designed to retrieve and display a list of instructors
who have conducted a number of lessons above the given threshold
\mintinline{sql}{threshold INT} within the current month.

The output found in table \ref{table:q3},
was retrieved when calling
\mintinline{sql}{fn_get_instructor_lesson_counts_above_threshold(2)}.

\begin{longlisting}
  \inputminted[
      label=Q3-Function,
      linenos=true,
      bgcolor=lightgray,
      firstline=144,
      lastline=163,
%        frame=single,
      fontsize=\footnotesize,
  ]{sql}{../../src/db/analytics/analytics.sql}
  \caption{
    Function that
list instructors who have given more than a specific number of
lessons during the current month.
    }
  \label{listing:q3}
\end{longlisting}

\begin{table}[h]
\centering
\begin{tabular}{|l|l|l|l|}
\hline
Instructor Id & First name & Last name   & No. of Lessons \\ \hline
25            & Anna       & Andersson   & 4              \\
26            & Björn      & Borg        & 4              \\
27            & Karin      & Karlsson    & 4              \\
28            & Göran      & Gustafsson  & 3              \\ \hline
\end{tabular}
\caption{Instructor Information}
\label{table:q3}
\end{table}

\clearpage
\subsection*{Fourth Query}
The SQL statement found in listing \ref{listing:q4} creates (or replaces) a view named
\mintinline{sql}{view_ensembles_next_week}.
This view is structured to provide information about ensemble sessions scheduled for the upcoming week.
In essence, this view provides a summary of all ensemble sessions scheduled for the next week,
including details about the day, genre, and availability of seats.
(A materialized view was considered -- see subsection \nameref{subsec:materializedview} in the \nameref{sec:discussion} section.)

The output of this view can be found in table \ref{table:q4}.

\begin{longlisting}
  \inputminted[
      label=Q4-VIEW,
      linenos=true,
      bgcolor=lightgray,
      firstline=169,
      lastline=184,
%        frame=single,
      fontsize=\footnotesize,
  ]{sql}{../../src/db/analytics/analytics.sql}
  \caption{
    Creates a view that 
show all ensembles held during the week ahead.
    }
  \label{listing:q4}
\end{longlisting}

\begin{table}[h]
\centering
\begin{tabular}{|l|l|l|}
\hline
Day  & Genre     & No. of Free Seats \\ \hline
Mon  & classical & Many Seats        \\ 
Wed  & classical & 1 or 2 Seats      \\ 
Mon  & rock      & Many Seats        \\ \hline
\end{tabular}
\caption{Concert Availability}
\label{table:q4}
\end{table}

\clearpage
\subsection*{Historical Data and denormalization}
In listing \ref{listing:hg}, one can find the
script designed to work with denormalized data layer optimized for OLAP (Online Analytical Processing) purposes.
It involves creating a table and a function to manage and analyze historical pricing data for lessons.
\begin{longlisting}
    \inputminted[
        label=OLAP,
        linenos=true,
        bgcolor=lightgray,
        firstline=187,
        lastline=282,
%        frame=single,
        fontsize=\footnotesize,
    ]{sql}{../../src/db/analytics/analytics.sql}
    \caption{
        Script for OLAP.
        It is designed to store historical lesson data,
        including lesson type, genre, instrument, price, and student details.
    }
    \label{listing:hg}
\end{longlisting}

In summary, this script sets up a system to store and analyze historical lesson pricing data,
combining information about the lessons and students.
It is designed for analytical purposes,
making it easier to review and understand pricing trends over time.
The approach is denormalized,
meaning it focuses on reducing the number of joins and simplifying queries at the cost of redundancy, which is typically well-suited for OLAP scenarios.

\section{Discussion}
\label{sec:discussion}
\subsection*{Subqueries vs joins}
As mentioned above, when it is necessary to retrieve data from multiple tables within a database,
there exist two primary alternatives: \emph{subqueries} and \emph{joins}.
Subqueries involve nesting a query within another, whereas join operations
combine rows from various tables, ''joining'' them through a shared column or a
specific condition.

As one can read from the output from EXPLAIN ANALYZE
(see listing \ref{listing:q2_explain}),
the subquery-view is a bit slower.
It is likely to be slower due to its more complex operations (like
nested loops and multiple aggregations), potentially less efficient data access
patterns, and possibly less effective use of indexes.
The difference in
execution time might not be significant for small datasets, but as data volume
increases, these factors can lead to more noticeable performance discrepancies.

\subsection*{The use of functions}
The project group chose to write a function for the third query.
According to the assignment, the output should list all instructors who
'' has given more than a specific number of lessons during the current month''.
Since neither views nor queries cannot accept parameters in a straight forward manner,
a function a threshold parameter.
This function executes a SQL query that joins three tables: session, instructor, and person
and filters the output accordingly.

\subsection*{Materialized View}
\label{subsec:materializedview}
In the context of query 4,
\mintinline{sql}{view_ensembles_next_week},
a regular view would ensure that the information about ensemble sessions is always current,
reflecting any new bookings or changes immediately.
A materialized view, on the other hand, would provide quicker access to this data,
especially if the underlying query is complex and the data does not change frequently.
However, one would need to manage the refreshes to ensure the data isn't too stale,
especially since it involves upcoming schedules that might change relatively often.
The choice between the two depends on the specific requirements of the application,
particularly regarding the balance between the need for up-to-date data and query performance.

For instance, if the application requires real-time accuracy,
such as in a dynamic booking system where session availability changes frequently,
a regular view might be more appropriate.
On the other hand, if the query is resource-intensive and the data changes infrequently or
if a slight delay in data freshness is acceptable, a materialized view could be more efficient and performant.

\subsection*{Denormalization}
Denormalization reduces the complexity of queries,
as data is available in a single table instead of being spread across multiple normalized tables.
Since the data is stored in a single table, read operations are faster,
which may be beneficial for generating reports and analytics.
The biggest advantage in this case is its suitability for historical data.
Historical data does not require frequent updates and is therefore
well-suited for denormalization as it minimizes the impact of one of denormalization's primary drawbacks:
namely the complexity of data updates.

However, denormalization may lead to data duplication,
with the same information (like lesson prices) repeated across multiple rows.
It will increase storage space, due to
data redundancy.
Denormalized databases can require significantly more storage space.
As mentioned, its primary drawback is maintenance complexity.
Any updates or changes
(though less frequent in historical data) can be more complex,
as they might need to be applied to multiple instances of the same data.

% \textbf{This chapter \textit{analysis} the result presented in the previous section.}

% Evaluate your solution according to the assessment criteria found in the assessment-criteria documents, which are found under the bullet \textit{In the Discussion chapter of your report...}, under each task on the Project page in Canvas. You do not have to cover all specified criteria.

\end{document}
